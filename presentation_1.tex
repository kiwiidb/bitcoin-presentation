%%%%%%%%%%%%%%%%%%%%%%%%%%%%%%%%%%%%%%%%%
% Beamer Presentation
% LaTeX Template
% Version 1.0 (10/11/12)
%
% This template has been downloaded from:
% http://www.LaTeXTemplates.com
%
% License:
% CC BY-NC-SA 3.0 (http://creativecommons.org/licenses/by-nc-sa/3.0/)
%
%%%%%%%%%%%%%%%%%%%%%%%%%%%%%%%%%%%%%%%%%

%----------------------------------------------------------------------------------------
%	PACKAGES AND THEMES
%----------------------------------------------------------------------------------------

\documentclass{beamer}

\mode<presentation> {

% The Beamer class comes with a number of default slide themes
% which change the colors and layouts of slides. Below this is a list
% of all the themes, uncomment each in turn to see what they look like.

%\usetheme{default}
%\usetheme{AnnArbor}
%\usetheme{Antibes}
%\usetheme{Bergen}
%\usetheme{Berkeley}
%\usetheme{Berlin}
%\usetheme{Boadilla}
%\usetheme{CambridgeUS}
%\usetheme{Copenhagen}
%\usetheme{Darmstadt}
%\usetheme{Dresden}
%\usetheme{Frankfurt}
%\usetheme{Goettingen}
%\usetheme{Hannover}
%\usetheme{Ilmenau}
%\usetheme{JuanLesPins}
%\usetheme{Luebeck}
\usetheme{Madrid}
%\usetheme{Malmoe}
%\usetheme{Marburg}
%\usetheme{Montpellier}
%\usetheme{PaloAlto}
%\usetheme{Pittsburgh}
%\usetheme{Rochester}
%\usetheme{Singapore}
%\usetheme{Szeged}
%\usetheme{Warsaw}

% As well as themes, the Beamer class has a number of color themes
% for any slide theme. Uncomment each of these in turn to see how it
% changes the colors of your current slide theme.

%\usecolortheme{albatross}
%\usecolortheme{beaver}
%\usecolortheme{beetle}
%\usecolortheme{crane}
%\usecolortheme{dolphin}
%\usecolortheme{dove}
%\usecolortheme{fly}
%\usecolortheme{lily}
%\usecolortheme{orchid}
%\usecolortheme{rose}
%\usecolortheme{seagull}
%\usecolortheme{seahorse}
%\usecolortheme{whale}
%\usecolortheme{wolverine}

%\setbeamertemplate{footline} % To remove the footer line in all slides uncomment this line
%\setbeamertemplate{footline}[page number] % To replace the footer line in all slides with a simple slide count uncomment this line

%\setbeamertemplate{navigation symbols}{} % To remove the navigation symbols from the bottom of all slides uncomment this line
}

\usepackage{graphicx} % Allows including images
\usepackage{booktabs} % Allows the use of \toprule, \midrule and \bottomrule in tables

%----------------------------------------------------------------------------------------
%	TITLE PAGE
%----------------------------------------------------------------------------------------

\title[Lightning Network]{Bitcoin Transactions and Lightning Network} % The short title appears at the bottom of every slide, the full title is only on the title page

\author{Kwinten De Backer} % Your name

\date{\today} % Date, can be changed to a custom date

\begin{document}

\begin{frame}
\titlepage % Print the title page as the first slide
\end{frame}

\begin{frame}
\frametitle{Overview} % Table of contents slide, comment this block out to remove it
\tableofcontents % Throughout your presentation, if you choose to use \section{} and \subsection{} commands, these will automatically be printed on this slide as an overview of your presentation
\end{frame}

%----------------------------------------------------------------------------------------
%	PRESENTATION SLIDES
%----------------------------------------------------------------------------------------

%------------------------------------------------
\section{Bitcoin Transactions} % Sections can be created in order to organize your presentation into discrete blocks, all sections and subsections are automatically printed in the table of contents as an overview of the talk
%------------------------------------------------

\begin{frame}
\frametitle{General format of a Bitcoin transaction}

\end{frame}

%------------------------------------------------
\begin{frame}
\frametitle{Inputs}
\end{frame}

\begin{frame}
\frametitle{Outputs}
\end{frame}

\begin{frame}
\frametitle{Verification}
\end{frame}

\begin{frame}
\frametitle{What does it mean to own Bitcoin?}


\end{frame}

%------------------------------------------------

\begin{frame}
\frametitle{Pay to public key hash}

\end{frame}

\begin{frame}
\frametitle{Pay to script hash}

\end{frame}
\section{The double spending problem}
\begin{frame}
\frametitle{Mining (short)}

\end{frame}
\section{Problems with scalability}
\begin{frame}
\frametitle{Transaction fees}

\end{frame}
\begin{frame}
\frametitle{Confirmation delay}

\end{frame}
\begin{frame}
\frametitle{Blockspace and decentralization}

\end{frame}
\section{The Lightning Network}
\begin{frame}
\frametitle{Lightning on a conceptual level}

\end{frame}
\begin{frame}
\frametitle{Payment channels}

\end{frame}
\begin{frame}
\frametitle{Payment channel network (naive)}

\end{frame}
\begin{frame}
\frametitle{Payment channel network (better)}

\end{frame}
%------------------------------------------------

\begin{frame}
\frametitle{Lightning problems and solutions}

\end{frame}
\section{The real mindblowing stuff}
\begin{frame}
\frametitle{Cross chain atomic swaps}

\end{frame}
\begin{frame}
\frametitle{Submarine swaps}

\end{frame}

\begin{frame}
\frametitle{Oracles}

\end{frame}
\begin{frame}
\frametitle{Channel Factories}

\end{frame}
%------------------------------------------------

\begin{frame}
\frametitle{References}
\footnotesize{
\begin{thebibliography}{99} % Beamer does not support BibTeX so references must be inserted manually as below
\bibitem[Smith, 2012]{p1} John Smith (2012)
\newblock Title of the publication
\newblock \emph{Journal Name} 12(3), 45 -- 678.
\end{thebibliography}
}
\end{frame}

%------------------------------------------------

\begin{frame}
\Huge{\centerline{The End}}
\end{frame}

%----------------------------------------------------------------------------------------

\end{document} 